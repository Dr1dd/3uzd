Antra objektinio programavimo užduotis. \subsection*{Reikiami failai}


\begin{DoxyItemize}
\item G\+NU compileriai\+: \href{http://www.mingw.org}{\texttt{ link\+: mingw}} \subsubsection*{Programos kompiliavimas ir paleidimas}
\end{DoxyItemize}

Turint reikiamus kompiliatorius vykdomi šie žingsniai\+: \href{https://imgur.com/a/yB0usfs}{\texttt{ Imgur linkas}} \section*{Releasai\+:}

\subsubsection*{\href{https://github.com/Dr1dd/2-u-d./releases/tag/v0.1}{\texttt{ 1 Releasas (v0.\+1)}}}

Programa idealizuota maždaug pagal reikalavimus, bet skaičių generavimas yra generuojamas pseudo generatoriumi. Tai bus ištaisyta versijoje v1.\+0 \subsubsection*{\href{https://github.com/Dr1dd/2-u-d./releases/tag/v0.2}{\texttt{ 2 Releasas (v0.\+2)}}}

Pridėtas pasirinkimas programoje tarp įvesties ir skaitymo iš failo. Be C masyvų. Rikiavimas veikia pagal užduoties \href{https://github.com/objprog/paskaitos2019/wiki/2-oji-u%C5%BEduotis\#reikalavimai-versijai-v02-terminas-2019-02-24-}{\texttt{ reikalavimus\+:}} \subsubsection*{\href{https://github.com/Dr1dd/2-u-d./releases/tag/v0.3}{\texttt{ 3 Releasas (v0.\+3)}}}

Programoje pradedamos naudoti struktūros ir header failas. Pradedamas naudoti išimčių valdymas (try-\/catch) programoje, kur vartotojo Integer įvestis gali būti per ilga (pvz.\+: skaičius 10$^\wedge$10) 
\begin{DoxyCode}{0}
\DoxyCodeLine{    while(valid)\{}
\DoxyCodeLine{        valid = false;}
\DoxyCodeLine{    try\{}
\DoxyCodeLine{        studSkaic = TikrintiSkaicius(tekstas);      }
\DoxyCodeLine{    \}}
\DoxyCodeLine{    catch(const std::out\_of\_range\& e)\{}
\DoxyCodeLine{        std::cout<< "Sis skaicius yra per didelis, bandykite dar karta" <<std::endl;}
\DoxyCodeLine{        valid = true;}
\DoxyCodeLine{        }
\DoxyCodeLine{    \}}
\DoxyCodeLine{\}}
\end{DoxyCode}
 \subsubsection*{\href{https://github.com/Dr1dd/2-u-d./releases/tag/v0.4}{\texttt{ 4 Releasas (v0.\+4)}}}

Programos veikimo laikai\+:

\tabulinesep=1mm
\begin{longtabu}spread 0pt [c]{*{2}{|X[-1]}|}
\hline
\PBS\centering \cellcolor{\tableheadbgcolor}\textbf{ Bandymas gen.  }&\PBS\centering \cellcolor{\tableheadbgcolor}\textbf{ Laikas   }\\\cline{1-2}
\endfirsthead
\hline
\endfoot
\hline
\PBS\centering \cellcolor{\tableheadbgcolor}\textbf{ Bandymas gen.  }&\PBS\centering \cellcolor{\tableheadbgcolor}\textbf{ Laikas   }\\\cline{1-2}
\endhead
\PBS\centering 1.  &15.\+64 s.   \\\cline{1-2}
\PBS\centering 2.  &11.\+35 s.   \\\cline{1-2}
\PBS\centering 3.  &13.\+61 s.   \\\cline{1-2}
\PBS\centering 4.  &10.\+91 s.   \\\cline{1-2}
\end{longtabu}



\begin{DoxyItemize}
\item \href{https://imgur.com/a/emVwq4E}{\texttt{ Failų generavimas (foto)}}
\end{DoxyItemize}

\tabulinesep=1mm
\begin{longtabu}spread 0pt [c]{*{2}{|X[-1]}|}
\hline
\PBS\centering \cellcolor{\tableheadbgcolor}\textbf{ Bandymas kursiokai.\+txt  }&\PBS\centering \cellcolor{\tableheadbgcolor}\textbf{ Laikas   }\\\cline{1-2}
\endfirsthead
\hline
\endfoot
\hline
\PBS\centering \cellcolor{\tableheadbgcolor}\textbf{ Bandymas kursiokai.\+txt  }&\PBS\centering \cellcolor{\tableheadbgcolor}\textbf{ Laikas   }\\\cline{1-2}
\endhead
\PBS\centering 1.  &2.\+57 s.   \\\cline{1-2}
\PBS\centering 2.  &2.\+8 s.   \\\cline{1-2}
\end{longtabu}



\begin{DoxyItemize}
\item \href{https://imgur.com/a/xGUrfVX}{\texttt{ Skaitymas iš kursiokai.\+txt (foto)}} \subsubsection*{\href{https://github.com/Dr1dd/2-u-d./releases/tag/v0.5}{\texttt{ 5 Releasas (v0.\+5)}}}
\end{DoxyItemize}

Nuotraukos su kiekvieno konteinerio generavimo laikais yra pačiame \href{https://github.com/Dr1dd/2-u-d./releases/tag/v0.5}{\texttt{ release.}}

\tabulinesep=1mm
\begin{longtabu}spread 0pt [c]{*{2}{|X[-1]}|}
\hline
\PBS\centering \cellcolor{\tableheadbgcolor}\textbf{ Bandymas (Skaitant iš sugeneruotų failų)  }&\PBS\centering \cellcolor{\tableheadbgcolor}\textbf{ Laikas   }\\\cline{1-2}
\endfirsthead
\hline
\endfoot
\hline
\PBS\centering \cellcolor{\tableheadbgcolor}\textbf{ Bandymas (Skaitant iš sugeneruotų failų)  }&\PBS\centering \cellcolor{\tableheadbgcolor}\textbf{ Laikas   }\\\cline{1-2}
\endhead
\PBS\centering vektoriai  &13.\+73 s.   \\\cline{1-2}
\PBS\centering listai  &14.\+23 s.   \\\cline{1-2}
\PBS\centering dekai  &13.\+70 s.   \\\cline{1-2}
\end{longtabu}






\tabulinesep=1mm
\begin{longtabu}spread 0pt [c]{*{2}{|X[-1]}|}
\hline
\PBS\centering \cellcolor{\tableheadbgcolor}\textbf{ Bandymas (Skaitant ir spausdinant iš kursiokai.\+txt)  }&\PBS\centering \cellcolor{\tableheadbgcolor}\textbf{ Laikas   }\\\cline{1-2}
\endfirsthead
\hline
\endfoot
\hline
\PBS\centering \cellcolor{\tableheadbgcolor}\textbf{ Bandymas (Skaitant ir spausdinant iš kursiokai.\+txt)  }&\PBS\centering \cellcolor{\tableheadbgcolor}\textbf{ Laikas   }\\\cline{1-2}
\endhead
\PBS\centering vektoriai  &0.\+04 s.   \\\cline{1-2}
\PBS\centering listai  &0.\+03 s.   \\\cline{1-2}
\PBS\centering dekai  &0.\+04 s.   \\\cline{1-2}
\end{longtabu}


Failai nėra generuojami. Taip pat rūšiuojami studentai nėra išsaugojami jokiame konteineryje. Tai bus padaryta versijoje v1.\+0.

Laikas skaičiuojamas tik skaitant iš failų (10-\/100000.\+txt ir kursiokai.\+txt) (pagal tam tikrą konteinerio tipą\+: vector, list ar deque) \subsubsection*{\href{https://github.com/Dr1dd/2-u-d./releases/tag/v1.0}{\texttt{ Paskutinis 6 Releasas (v1.\+0)}}}

Paskutinė programos versija reikalavo paanalizuoti programos veikimo laikus pagal naudojamus konteinerius ir tam tikrus algoritmus, kurie galimai paspartintų programos veikimą.

Užduotyje nurodytų studentų rūšiavimą reikėjo perteikti trimis skirtingais konteineriais\+: vektoriais, dekais ir listais. Taip pat rūšiavima teko realizuoti pagal dvi strategijas\+:


\begin{DoxyEnumerate}
\item Susikuriant dar 2 naujus tokio pačio tipo konteinerius ir į juos surūšiuoti studentus, kurie \char`\"{}išlaikė\char`\"{} arba \char`\"{}neišlaikė\char`\"{}.
\item Susikuriant 1 naują tokio pat tipo konteinerį ir į jį sukeliant studentus, kurie \char`\"{}neišlaikė\char`\"{}, na ir iš pagrindinio konteinerio ištrinant tuos pačius studentus.
\end{DoxyEnumerate}

{\bfseries{Programos laiko skaičiavimai\+:}}

\tabulinesep=1mm
\begin{longtabu}spread 0pt [c]{*{2}{|X[-1]}|}
\hline
\PBS\centering \cellcolor{\tableheadbgcolor}\textbf{ Vektoriai 1 strat. (\mbox{\hyperlink{class_studentai}{Studentai}})  }&\PBS\centering \cellcolor{\tableheadbgcolor}\textbf{ Laikas   }\\\cline{1-2}
\endfirsthead
\hline
\endfoot
\hline
\PBS\centering \cellcolor{\tableheadbgcolor}\textbf{ Vektoriai 1 strat. (\mbox{\hyperlink{class_studentai}{Studentai}})  }&\PBS\centering \cellcolor{\tableheadbgcolor}\textbf{ Laikas   }\\\cline{1-2}
\endhead
10  &0.\+002179 s.   \\\cline{1-2}
100  &0.\+076071 s.   \\\cline{1-2}
1000  &0.\+044098 s.   \\\cline{1-2}
10000  &0.\+165092 s.   \\\cline{1-2}
100000  &1.\+42633 s.   \\\cline{1-2}
\end{longtabu}


\tabulinesep=1mm
\begin{longtabu}spread 0pt [c]{*{2}{|X[-1]}|}
\hline
\PBS\centering \cellcolor{\tableheadbgcolor}\textbf{ Vektoriai 2 strat. (\mbox{\hyperlink{class_studentai}{Studentai}})  }&\PBS\centering \cellcolor{\tableheadbgcolor}\textbf{ Laikas   }\\\cline{1-2}
\endfirsthead
\hline
\endfoot
\hline
\PBS\centering \cellcolor{\tableheadbgcolor}\textbf{ Vektoriai 2 strat. (\mbox{\hyperlink{class_studentai}{Studentai}})  }&\PBS\centering \cellcolor{\tableheadbgcolor}\textbf{ Laikas   }\\\cline{1-2}
\endhead
10  &0.\+00248 s.   \\\cline{1-2}
100  &0.\+008927 s.   \\\cline{1-2}
1000  &0.\+289948 s.   \\\cline{1-2}
10000  &25.\+7449 s.   \\\cline{1-2}
100000  &∞   \\\cline{1-2}
\end{longtabu}






\tabulinesep=1mm
\begin{longtabu}spread 0pt [c]{*{2}{|X[-1]}|}
\hline
\PBS\centering \cellcolor{\tableheadbgcolor}\textbf{ Listai 1 strat. (\mbox{\hyperlink{class_studentai}{Studentai}})  }&\PBS\centering \cellcolor{\tableheadbgcolor}\textbf{ Laikas   }\\\cline{1-2}
\endfirsthead
\hline
\endfoot
\hline
\PBS\centering \cellcolor{\tableheadbgcolor}\textbf{ Listai 1 strat. (\mbox{\hyperlink{class_studentai}{Studentai}})  }&\PBS\centering \cellcolor{\tableheadbgcolor}\textbf{ Laikas   }\\\cline{1-2}
\endhead
10  &0.\+00234 s.   \\\cline{1-2}
100  &0.\+006944 s.   \\\cline{1-2}
1000  &0.\+02109 s.   \\\cline{1-2}
10000  &0.\+13012 s.   \\\cline{1-2}
100000  &1.\+38188 s.   \\\cline{1-2}
\end{longtabu}


\tabulinesep=1mm
\begin{longtabu}spread 0pt [c]{*{2}{|X[-1]}|}
\hline
\PBS\centering \cellcolor{\tableheadbgcolor}\textbf{ Listai 2 strat. (\mbox{\hyperlink{class_studentai}{Studentai}})  }&\PBS\centering \cellcolor{\tableheadbgcolor}\textbf{ Laikas   }\\\cline{1-2}
\endfirsthead
\hline
\endfoot
\hline
\PBS\centering \cellcolor{\tableheadbgcolor}\textbf{ Listai 2 strat. (\mbox{\hyperlink{class_studentai}{Studentai}})  }&\PBS\centering \cellcolor{\tableheadbgcolor}\textbf{ Laikas   }\\\cline{1-2}
\endhead
10  &0.\+244209 s.   \\\cline{1-2}
100  &0.\+005561 s.   \\\cline{1-2}
1000  &0.\+018242 s.   \\\cline{1-2}
10000  &0.\+127754 s.   \\\cline{1-2}
100000  &1.\+2391 s.   \\\cline{1-2}
\end{longtabu}






\tabulinesep=1mm
\begin{longtabu}spread 0pt [c]{*{2}{|X[-1]}|}
\hline
\PBS\centering \cellcolor{\tableheadbgcolor}\textbf{ Dekai 1 strat. (\mbox{\hyperlink{class_studentai}{Studentai}})  }&\PBS\centering \cellcolor{\tableheadbgcolor}\textbf{ Laikas   }\\\cline{1-2}
\endfirsthead
\hline
\endfoot
\hline
\PBS\centering \cellcolor{\tableheadbgcolor}\textbf{ Dekai 1 strat. (\mbox{\hyperlink{class_studentai}{Studentai}})  }&\PBS\centering \cellcolor{\tableheadbgcolor}\textbf{ Laikas   }\\\cline{1-2}
\endhead
10  &0.\+003966 s.   \\\cline{1-2}
100  &0.\+007231 s.   \\\cline{1-2}
1000  &0.\+018848 s.   \\\cline{1-2}
10000  &0.\+136482 s.   \\\cline{1-2}
100000  &1.\+34894 s.   \\\cline{1-2}
\end{longtabu}


\tabulinesep=1mm
\begin{longtabu}spread 0pt [c]{*{2}{|X[-1]}|}
\hline
\PBS\centering \cellcolor{\tableheadbgcolor}\textbf{ Dekai 2 strat. (\mbox{\hyperlink{class_studentai}{Studentai}})  }&\PBS\centering \cellcolor{\tableheadbgcolor}\textbf{ Laikas   }\\\cline{1-2}
\endfirsthead
\hline
\endfoot
\hline
\PBS\centering \cellcolor{\tableheadbgcolor}\textbf{ Dekai 2 strat. (\mbox{\hyperlink{class_studentai}{Studentai}})  }&\PBS\centering \cellcolor{\tableheadbgcolor}\textbf{ Laikas   }\\\cline{1-2}
\endhead
10  &0.\+003471 s.   \\\cline{1-2}
100  &0.\+006946 s.   \\\cline{1-2}
1000  &0.\+06399 s.   \\\cline{1-2}
10000  &4.\+53074 s.   \\\cline{1-2}
100000  &∞   \\\cline{1-2}
\end{longtabu}


{\bfseries{Su optimizavimais\+:}}

\tabulinesep=1mm
\begin{longtabu}spread 0pt [c]{*{2}{|X[-1]}|}
\hline
\PBS\centering \cellcolor{\tableheadbgcolor}\textbf{ Vektoriai 1 strat. + copyif (\mbox{\hyperlink{class_studentai}{Studentai}})  }&\PBS\centering \cellcolor{\tableheadbgcolor}\textbf{ Laikas   }\\\cline{1-2}
\endfirsthead
\hline
\endfoot
\hline
\PBS\centering \cellcolor{\tableheadbgcolor}\textbf{ Vektoriai 1 strat. + copyif (\mbox{\hyperlink{class_studentai}{Studentai}})  }&\PBS\centering \cellcolor{\tableheadbgcolor}\textbf{ Laikas   }\\\cline{1-2}
\endhead
10  &0.\+002482 s.   \\\cline{1-2}
100  &0.\+007764 s.   \\\cline{1-2}
1000  &0.\+020686 s.   \\\cline{1-2}
10000  &0.\+162558 s.   \\\cline{1-2}
100000  &1.\+61132 s.   \\\cline{1-2}
\end{longtabu}






\tabulinesep=1mm
\begin{longtabu}spread 0pt [c]{*{2}{|X[-1]}|}
\hline
\PBS\centering \cellcolor{\tableheadbgcolor}\textbf{ Vektoriai 2 strat. + removeif (\mbox{\hyperlink{class_studentai}{Studentai}})  }&\PBS\centering \cellcolor{\tableheadbgcolor}\textbf{ Laikas   }\\\cline{1-2}
\endfirsthead
\hline
\endfoot
\hline
\PBS\centering \cellcolor{\tableheadbgcolor}\textbf{ Vektoriai 2 strat. + removeif (\mbox{\hyperlink{class_studentai}{Studentai}})  }&\PBS\centering \cellcolor{\tableheadbgcolor}\textbf{ Laikas   }\\\cline{1-2}
\endhead
10  &0.\+003924 s.   \\\cline{1-2}
100  &0.\+007018 s.   \\\cline{1-2}
1000  &0.\+026792 s.   \\\cline{1-2}
10000  &0.\+216071 s.   \\\cline{1-2}
100000  &1.\+60784 s.   \\\cline{1-2}
\end{longtabu}


\tabulinesep=1mm
\begin{longtabu}spread 0pt [c]{*{2}{|X[-1]}|}
\hline
\PBS\centering \cellcolor{\tableheadbgcolor}\textbf{ Vektoriai 2 strat. + removeif + copyif (\mbox{\hyperlink{class_studentai}{Studentai}})  }&\PBS\centering \cellcolor{\tableheadbgcolor}\textbf{ Laikas   }\\\cline{1-2}
\endfirsthead
\hline
\endfoot
\hline
\PBS\centering \cellcolor{\tableheadbgcolor}\textbf{ Vektoriai 2 strat. + removeif + copyif (\mbox{\hyperlink{class_studentai}{Studentai}})  }&\PBS\centering \cellcolor{\tableheadbgcolor}\textbf{ Laikas   }\\\cline{1-2}
\endhead
10  &0.\+002486 s.   \\\cline{1-2}
100  &0.\+007804 s.   \\\cline{1-2}
1000  &0.\+02193 s.   \\\cline{1-2}
10000  &0.\+153789 s.   \\\cline{1-2}
100000  &1.\+4865 s.   \\\cline{1-2}
\end{longtabu}



\begin{DoxyItemize}
\item Funkcijos naudojamos programos efektyvumui pagerinti\+: \href{https://imgur.com/a/ZUuqimf}{\texttt{ imgur}}
\end{DoxyItemize}

Greičiausias gautas laikas buvo gautas naudojant list\textquotesingle{}us ir 2 studentų rūšiavimo strategiją.

Nuotraukos su programos laikais\+: \href{https://imgur.com/a/WJtahba}{\texttt{ vektoriai}}, \href{https://imgur.com/a/adZMfZU}{\texttt{ listai}}, \href{https://imgur.com/a/hKofp2w}{\texttt{ dekai}} \subsubsection*{\href{https://github.com/Dr1dd/2-u-d./releases/tag/v1.01}{\texttt{ Papildomas Releasas (v1.\+01)}}}

Papildoma užduotis reikalauja dar vieno studentų rūšiavimo būdo -\/$>$ push backinti mažiau nei 5 balus gavusius studentus į naują vector konteinerį (šiuo atveju pav. neismoko) ir įterpti daugiau arba 5 gavusius studentus į tą patį pradinį buvusį vector konteinerį (mano atveju Studentu\+Info).

Buvo sukurtos naujos funkcijos\+: \mbox{\hyperlink{funkcijos_8h_a5437a7161f574630674c925f625fa6a7}{rask\+Minkstus()}}; \mbox{\hyperlink{funkcijos_8h_add6c99ace6c7d15928dd5a9244f9a445}{iterpk\+Kietus()}}; rask\+Minkstus\+Deque(); iterpk\+Kietus\+Deque();

\tabulinesep=1mm
\begin{longtabu}spread 0pt [c]{*{2}{|X[-1]}|}
\hline
\PBS\centering \cellcolor{\tableheadbgcolor}\textbf{ Vektoriai papildoma strat. \char`\"{}minkštų\char`\"{} rūšiavimas  }&\PBS\centering \cellcolor{\tableheadbgcolor}\textbf{ Laikas   }\\\cline{1-2}
\endfirsthead
\hline
\endfoot
\hline
\PBS\centering \cellcolor{\tableheadbgcolor}\textbf{ Vektoriai papildoma strat. \char`\"{}minkštų\char`\"{} rūšiavimas  }&\PBS\centering \cellcolor{\tableheadbgcolor}\textbf{ Laikas   }\\\cline{1-2}
\endhead
10  &0.\+003027 s.   \\\cline{1-2}
100  &0.\+005021 s.   \\\cline{1-2}
1000  &0.\+01382 s.   \\\cline{1-2}
10000  &0.\+067877 s.   \\\cline{1-2}
100000  &0.\+599438 s.   \\\cline{1-2}
\end{longtabu}


\tabulinesep=1mm
\begin{longtabu}spread 0pt [c]{*{2}{|X[-1]}|}
\hline
\PBS\centering \cellcolor{\tableheadbgcolor}\textbf{ Vektoriai papildoma strat. \char`\"{}kietų\char`\"{} rūšiavimas  }&\PBS\centering \cellcolor{\tableheadbgcolor}\textbf{ Laikas   }\\\cline{1-2}
\endfirsthead
\hline
\endfoot
\hline
\PBS\centering \cellcolor{\tableheadbgcolor}\textbf{ Vektoriai papildoma strat. \char`\"{}kietų\char`\"{} rūšiavimas  }&\PBS\centering \cellcolor{\tableheadbgcolor}\textbf{ Laikas   }\\\cline{1-2}
\endhead
10  &0.\+00303 s.   \\\cline{1-2}
100  &0.\+009994 s.   \\\cline{1-2}
1000  &0.\+374872 s.   \\\cline{1-2}
10000  &39.\+9822 s.   \\\cline{1-2}
100000  &+600 s.   \\\cline{1-2}
\end{longtabu}


Šiuo būdu rūšiuojant studentus \char`\"{}kieti\char`\"{} studentai yra rūšiuojami ytin ilgai.





\tabulinesep=1mm
\begin{longtabu}spread 0pt [c]{*{2}{|X[-1]}|}
\hline
\PBS\centering \cellcolor{\tableheadbgcolor}\textbf{ Dekai papildoma strat. \char`\"{}minkštų\char`\"{} rūšiavimas  }&\PBS\centering \cellcolor{\tableheadbgcolor}\textbf{ Laikas   }\\\cline{1-2}
\endfirsthead
\hline
\endfoot
\hline
\PBS\centering \cellcolor{\tableheadbgcolor}\textbf{ Dekai papildoma strat. \char`\"{}minkštų\char`\"{} rūšiavimas  }&\PBS\centering \cellcolor{\tableheadbgcolor}\textbf{ Laikas   }\\\cline{1-2}
\endhead
10  &0.\+004987 s.   \\\cline{1-2}
100  &0.\+005988 s.   \\\cline{1-2}
1000  &0.\+011147 s.   \\\cline{1-2}
10000  &0.\+071503 s.   \\\cline{1-2}
100000  &0.\+470746 s.   \\\cline{1-2}
\end{longtabu}


\tabulinesep=1mm
\begin{longtabu}spread 0pt [c]{*{2}{|X[-1]}|}
\hline
\PBS\centering \cellcolor{\tableheadbgcolor}\textbf{ Dekai papildoma strat. \char`\"{}kietų\char`\"{} rūšiavimas  }&\PBS\centering \cellcolor{\tableheadbgcolor}\textbf{ Laikas   }\\\cline{1-2}
\endfirsthead
\hline
\endfoot
\hline
\PBS\centering \cellcolor{\tableheadbgcolor}\textbf{ Dekai papildoma strat. \char`\"{}kietų\char`\"{} rūšiavimas  }&\PBS\centering \cellcolor{\tableheadbgcolor}\textbf{ Laikas   }\\\cline{1-2}
\endhead
10  &0.\+002991 s.   \\\cline{1-2}
100  &0.\+005986 s.   \\\cline{1-2}
1000  &0.\+014964 s.   \\\cline{1-2}
10000  &0.\+109712 s.   \\\cline{1-2}
100000  &0.\+80394 s.   \\\cline{1-2}
\end{longtabu}


Šiuo būdu rūšiuojant studentus, rūšiavimo laikas žymiai pagreitėja palyginus su vektoriais. Taip pat \char`\"{}minkštų\char`\"{} studentų rūšiavimo laikas yra trumpesnis nei \char`\"{}kietų\char`\"{}. Insert tipo rūšiavimas labiau tinka dekų konteineriams, nes į juos galima tiesiog \char`\"{}push front\textquotesingle{}inti\char`\"{}.

Nuotraukos su keliais bandymais\+: \href{https://imgur.com/a/kNWviiv}{\texttt{ imgur link}}

\section*{3 užduotis}

\subsection*{Releasai\+:}

\subsubsection*{\href{https://github.com/Dr1dd/3uzd/releases/tag/v1.1}{\texttt{ 1 Releasas (v1.\+1)}}}

Gauta užduotis buvo pakeisti stuktūrą {\ttfamily \mbox{\hyperlink{class_studentai}{Studentai}}} į klasę {\ttfamily \mbox{\hyperlink{class_studentai}{Studentai}}}.

Šią užduotį pavyko realizuoti naudojant get\textquotesingle{}terius ir set\textquotesingle{}terius. P\+VZ.\+: 
\begin{DoxyCode}{0}
\DoxyCodeLine{void setLname(std::string lname\_);}
\DoxyCodeLine{inline std::string pavarde() const \{ return lname; \}}
\end{DoxyCode}
 Užduotis reikalavo patikrinti programos veikimo laiką su struktūra ir su klase.

Rezultatai\+:

\tabulinesep=1mm
\begin{longtabu}spread 0pt [c]{*{3}{|X[-1]}|}
\hline
\PBS\centering \cellcolor{\tableheadbgcolor}\textbf{ Studentų skaičius  }&\PBS\centering \cellcolor{\tableheadbgcolor}\textbf{ Laikas 1 strategijos su struktūra  }&\PBS\centering \cellcolor{\tableheadbgcolor}\textbf{ Laikas 1 strategijos su klase   }\\\cline{1-3}
\endfirsthead
\hline
\endfoot
\hline
\PBS\centering \cellcolor{\tableheadbgcolor}\textbf{ Studentų skaičius  }&\PBS\centering \cellcolor{\tableheadbgcolor}\textbf{ Laikas 1 strategijos su struktūra  }&\PBS\centering \cellcolor{\tableheadbgcolor}\textbf{ Laikas 1 strategijos su klase   }\\\cline{1-3}
\endhead
10  &0.\+00403 s.  &0.\+007812 s.   \\\cline{1-3}
100  &0.\+0117 s.  &0.\+00757 s.   \\\cline{1-3}
1000  &0.\+01955 s.  &0.\+0196 s.   \\\cline{1-3}
10000  &0.\+10399 s.  &0.\+10800 s.   \\\cline{1-3}
100000  &0.\+8896 s.  &0.\+9760 s.   \\\cline{1-3}
\end{longtabu}


\tabulinesep=1mm
\begin{longtabu}spread 0pt [c]{*{3}{|X[-1]}|}
\hline
\PBS\centering \cellcolor{\tableheadbgcolor}\textbf{ Studentų skaičius  }&\PBS\centering \cellcolor{\tableheadbgcolor}\textbf{ Laikas 2 strategijos su struktūra  }&\PBS\centering \cellcolor{\tableheadbgcolor}\textbf{ Laikas 2 strategijos su klase   }\\\cline{1-3}
\endfirsthead
\hline
\endfoot
\hline
\PBS\centering \cellcolor{\tableheadbgcolor}\textbf{ Studentų skaičius  }&\PBS\centering \cellcolor{\tableheadbgcolor}\textbf{ Laikas 2 strategijos su struktūra  }&\PBS\centering \cellcolor{\tableheadbgcolor}\textbf{ Laikas 2 strategijos su klase   }\\\cline{1-3}
\endhead
10  &0.\+00404 s.  &0.\+01158 s.   \\\cline{1-3}
100  &0.\+007588 s.  &0.\+008025 s.   \\\cline{1-3}
1000  &0.\+01757 s.  &0.\+050232 s.   \\\cline{1-3}
10000  &0.\+10755 s.  &0.\+1119 s.   \\\cline{1-3}
100000  &0.\+98011 s.  &0.\+9739 s.   \\\cline{1-3}
\end{longtabu}


Programa veikia panašiu/vienodu greičiu.

Programos studentų rūšiavimo laikai naudojantis optimizavimo flag\textquotesingle{}us\+:

\tabulinesep=1mm
\begin{longtabu}spread 0pt [c]{*{2}{|X[-1]}|}
\hline
\PBS\centering \cellcolor{\tableheadbgcolor}\textbf{ Studentų skaičius  }&\PBS\centering \cellcolor{\tableheadbgcolor}\textbf{ Laikas be flag\textquotesingle{}ų   }\\\cline{1-2}
\endfirsthead
\hline
\endfoot
\hline
\PBS\centering \cellcolor{\tableheadbgcolor}\textbf{ Studentų skaičius  }&\PBS\centering \cellcolor{\tableheadbgcolor}\textbf{ Laikas be flag\textquotesingle{}ų   }\\\cline{1-2}
\endhead
10  &0.\+007947 s.   \\\cline{1-2}
100  &0.\+1673 s.   \\\cline{1-2}
1000  &0.\+0459 s.   \\\cline{1-2}
10000  &0.\+25997 s.   \\\cline{1-2}
100000  &1.\+0003 s.   \\\cline{1-2}
\end{longtabu}


\tabulinesep=1mm
\begin{longtabu}spread 0pt [c]{*{2}{|X[-1]}|}
\hline
\PBS\centering \cellcolor{\tableheadbgcolor}\textbf{ Studentų skaičius  }&\PBS\centering \cellcolor{\tableheadbgcolor}\textbf{ Laikas su -\/O1 flag\textquotesingle{}u   }\\\cline{1-2}
\endfirsthead
\hline
\endfoot
\hline
\PBS\centering \cellcolor{\tableheadbgcolor}\textbf{ Studentų skaičius  }&\PBS\centering \cellcolor{\tableheadbgcolor}\textbf{ Laikas su -\/O1 flag\textquotesingle{}u   }\\\cline{1-2}
\endhead
10  &0.\+007999 s.   \\\cline{1-2}
100  &0.\+01201 s.   \\\cline{1-2}
1000  &0.\+036 s.   \\\cline{1-2}
10000  &0.\+1115 s.   \\\cline{1-2}
100000  &0.\+951 s.   \\\cline{1-2}
\end{longtabu}


\tabulinesep=1mm
\begin{longtabu}spread 0pt [c]{*{2}{|X[-1]}|}
\hline
\PBS\centering \cellcolor{\tableheadbgcolor}\textbf{ Studentų skaičius  }&\PBS\centering \cellcolor{\tableheadbgcolor}\textbf{ Laikas su -\/O2 flag\textquotesingle{}u   }\\\cline{1-2}
\endfirsthead
\hline
\endfoot
\hline
\PBS\centering \cellcolor{\tableheadbgcolor}\textbf{ Studentų skaičius  }&\PBS\centering \cellcolor{\tableheadbgcolor}\textbf{ Laikas su -\/O2 flag\textquotesingle{}u   }\\\cline{1-2}
\endhead
10  &0.\+00594 s.   \\\cline{1-2}
100  &0.\+008034 s.   \\\cline{1-2}
1000  &0.\+02 s.   \\\cline{1-2}
10000  &0.\+1116 s.   \\\cline{1-2}
100000  &1.\+0226 s.   \\\cline{1-2}
\end{longtabu}


\tabulinesep=1mm
\begin{longtabu}spread 0pt [c]{*{2}{|X[-1]}|}
\hline
\PBS\centering \cellcolor{\tableheadbgcolor}\textbf{ Studentų skaičius  }&\PBS\centering \cellcolor{\tableheadbgcolor}\textbf{ Laikas su -\/O3 flag\textquotesingle{}u   }\\\cline{1-2}
\endfirsthead
\hline
\endfoot
\hline
\PBS\centering \cellcolor{\tableheadbgcolor}\textbf{ Studentų skaičius  }&\PBS\centering \cellcolor{\tableheadbgcolor}\textbf{ Laikas su -\/O3 flag\textquotesingle{}u   }\\\cline{1-2}
\endhead
10  &0.\+0116 s.   \\\cline{1-2}
100  &0.\+0356 s.   \\\cline{1-2}
1000  &0.\+016 s.   \\\cline{1-2}
10000  &0.\+273 s.   \\\cline{1-2}
100000  &0.\+956 s.   \\\cline{1-2}
\end{longtabu}


Kaip matome, keičiant optimizavimo flag\textquotesingle{}us, programos studentų rūšiavimo laikas beveik arba visiškai nesikeičia.

\subsubsection*{\href{https://github.com/Dr1dd/3uzd/releases/tag/v1.2}{\texttt{ 2 Releasas (v1.\+2)}}}

Gauta užduotis reikalavo realizuoti galimus programai papildomus operatorius. \char`\"{}\+Overload\textquotesingle{}inti\char`\"{} operatoriai\+:

{\ttfamily $>$} {\ttfamily $>$$>$} {\ttfamily $<$$<$} {\ttfamily =} {\ttfamily $<$=} {\ttfamily /} {\ttfamily +=} {\ttfamily ==}

Pvz.\+:



\subsubsection*{\href{https://github.com/Dr1dd/3uzd/releases/tag/v1.5}{\texttt{ 3 Releasas (v1.\+5)}}}

Ši užduotis reikalavo iš vienos turimos klasės padaryti dvi\+: bazinę (base) ir išvestinę (derived) klases. Išvestinė klasė paveldi tam tikrus kintamuosius ar funkcijas iš bazinės klasės. Šiuo atveju programoje buvo sukurti {\ttfamily Žmogaus} ir {\ttfamily Studentų} klasės. pvz.\+: Zmogaus bazinė klasė 
\begin{DoxyCode}{0}
\DoxyCodeLine{class Zmogus\{}
\DoxyCodeLine{    public:}
\DoxyCodeLine{void setFname(std::string fname\_)\{}
\DoxyCodeLine{    fname = fname\_;}
\DoxyCodeLine{\}}
\DoxyCodeLine{void setLname(std::string lname\_)\{}
\DoxyCodeLine{    lname = lname\_;}
\DoxyCodeLine{\}}
\DoxyCodeLine{virtual void setegzRez(double egzGalutinis) = 0;}
\DoxyCodeLine{    protected:}
\DoxyCodeLine{        std::string fname;}
\DoxyCodeLine{        std::string lname;}
\DoxyCodeLine{        double ugis;}
\DoxyCodeLine{        bool lytis; // pvz 1 - vyras 0 - moteris}
\DoxyCodeLine{\};}
\end{DoxyCode}
 Padaryta {\ttfamily virtual} funkcija reiškia, jog šios klasės objektų negalima padaryti. Jeigu virtual funkcija egzistuoja bazinėje klasėje, tai reiškia, kad šios klasės išvestinėse klasėse turės būtinai būti ta pati (pavadinimas toks pat) tik ne virtual funkcija.

Išvestinė klasė {\ttfamily \mbox{\hyperlink{class_studentai}{Studentai}}}\+:


\begin{DoxyCode}{0}
\DoxyCodeLine{class Studentai: public Zmogus\{}
\DoxyCodeLine{}
\DoxyCodeLine{\}}
\end{DoxyCode}
 Taip \mbox{\hyperlink{class_studentai}{Studentai}} klasė paveldi {\ttfamily Vardus} ir {\ttfamily Pavardes} iš {\ttfamily Žmogaus} klasės. 